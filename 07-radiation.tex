% !TEX root = 5b-main.tex

\chapter{Electromagnetic Radiation}

\section{Purpose}
	This laboratory exercise will allow you to explore the wave and particle natures of electromagnetic radiation.  You will be able to manipulate various electromagnetic sources of varying wavelengths from the microwave (long wavelength) to X-rays (short wavelengths).  
	You will also explore some of the properties of atoms, particularly how their electrons absorb and emit electromagnetic photons.  You will use your observations to calculate the physical properties of various forms of radiation.

\section{Preparation}
	You should review the electromagnetic spectrum in Section 23.3 and polarization in Section 23.10 of your textbook.  You will also need to read and understand the sections on the Photoelectric effect (28.1) and Line Spectra (28.2).  Pay special attention to how we detect and measure different types of radiation.

\paragraph{Short quiz}
  Be prepared to take a short quiz for which you need to be familiar with the major categories of electromagnetic waves\sidenote{Refer to Figure 23.3 in your textbook} and their properties such as the shape of an electromagnetic wave and its approximate wavelength. 
\section{General Information}

Your instructor will explain the operation of the numerous apparatuses in the room.  There is one station for each activity so you will circulate about the room to perform this lab.  You may perform the activities in any order.
Your instructor will also summarize the wave and photon characteristics of electromagnetic radiation.

\section{Exploring EM radiation}
Perform each of the following activities.  Be certain to keep good records of your observations and to answer the specific questions posed.  Note anything that you observe in addition to what is mentioned in this write up.  Such observations may be at least as important as those you are asked to make.

\subsection{Microwaves}
Microwaves are the highest-frequency electromagnetic waves that can be produced by currents in circuits.  They generally have a frequency that is the same as the oscillation frequency of the circuit.

\begin{table}
\centering
\begin{tabular}{l l l}\toprule
\# & $x$(cm) & Signal\\
\midrule
1.	&	2.84 &	max\\
2.	&	3.59 &	min\\
3.	&	4.31	 &	max	\\
4.	&	5.09 &	min\\
5.	&	5.86 &	max\\
6.	&	6.57	 &	min\\
%7.	&	7.36 &	max\\
%8.	&	8.14 &	min\\
%9.	&	8.85 &	max\\
%10.	& 	9.55 &	min\\
\bottomrule
\end{tabular}
\label{t:microwavesignal}
\caption[Microwave Peak Signal][-2em]{The values shown here display the results from a microwave setup. The displacement values, x, are measured along a straight line between the microwave transmitter and receiver. The signal values represent the locations where the voltage strength reaches a minima or maxima. For example, the value x = 2.84 cm indicates the location where the voltage reading is a maxima whereas the location x = 3.59 cm represents a location where the voltage reading is a minima. From these values you can determine the wavelength.}
\end{table}

\begin{enumerate}
	\item Sketch the microwave apparatus, labeling the major components.  Note that the only evidence for the microwaves is given by the meter connected to a detection device.
	\item Can you feel the microwaves?  Why don't they cook you like a microwave oven would\sidenote{\textsc{Hint:} Most home microwave ovens use approximately 1200 Watts.}?
	\item Measure the distance between the emitting horn and the detecting horn.  Now, slowly vary the distance until the detection meter reaches a minimum or maximum.  Record the new distance.  Find as many minima and maxima as you can in this manner.
	\item Calculate the frequency $f$ of the microwaves assuming they travel at the speed of light ($c=3\times10^{8}$ meters / second)\sidenote{Recall that the wavelength $\lambda$ is related to frequency and speed by \[\lambda=\frac{c}{f}\]}.  Don't forget that 1 wavelength is minimum to minimum or maximum to maximum.
	\item Compare your results to those given in \tref{t:microwavesignal}. How far off from the expected values of frequency and wavelength were you?  Does your lambda calculated using the minima agree with the lambda calculated using the maxima?
	\item Now find a maximum point for microwave strength.  Once you find the maximum, record the distance and the signal strength.  Now, insert a polarizing grid between the feedhorns.  Slowly rotate it until the signal strength is maximized again.  How much signal is lost in the grid?  
	\item Now rotate the grid until your signal is minimized.  How much residual signal still gets through the grid?  Does the signal lost in the previous step match the residual signal in this step?  
	\item Polarization fraction is calculated as \[\textrm{Pol. Frac.}=\frac{\textrm{Max w/polarizer}}{\textrm{Max w/o polarizer}}\] Calculate the polarization fraction for our microwave system.
\end{enumerate}
	
\subsection{Infrared (IR)}
Infrared radiation is usually produced by thermal motion and the vibration and rotation of atoms and molecules.  Its frequencies overlap with the upper end of the microwave range and extend to the lower end of the visible range hence the name infrared or ``below red.''
\begin{enumerate}
	\item Can you see infrared radiation with your eyes?  How do you detect it in this exercise?  
	\item Manipulate the reflectors to determine whether infrared radiation and visible light follow the same laws of reflection.  
	\item Perform the experiment using the large lens.  Do your results support the hypothesis that infrared and visible light behave similarly?
\end{enumerate}
	
\subsection{Visible Light} 
Visible light is defined to be the part of the electromagnetic spectrum to which the eye normally responds, producing nerve signals.  Visible light can be produced by a wide variety of processes.

	\begin{quote}
	\vspace{1em}
	\hrule
	\textsc{Warning:} Our lasers emit narrow beams of light and it is not dangerous to look at the reflected light.  However, you should never look directly at a laser beam or point it toward someone's head as direct exposure can cause temporary loss of vision.
	\hrule
	\end{quote}
	
\begin{enumerate}
	\item \textbf{Visible Laser Light:}\sidenote{The name "LASER" is an acronym for Light Amplification by Stimulated Emission of Radiation.  Lasers emit electromagnetic waves that have a very pure frequency and wavelength and which are coherent (all the waves are in phase). } 
	
	\begin{enumerate}
		\item Pass the laser light through a narrow slit and describe what you observe.  How are your observations consistent with the wave nature of electromagnetic radiation?
		\item Observe the hologram image.  How can you tell that it is a true three-dimensional image?
		\item List several applications of lasers.  Explain how each application is related to the pure frequency and coherent nature of laser output.
	\end{enumerate}
	\clearpage
	\item \textbf{Polarization of Visible Light:}
	\begin{enumerate}
		\item Devise an experiment that demonstrates the polarization of visible light.\sidenote{See Section 23.10 in your textbook for example of polarization} Two methods are effective and should be explored. The first is the passage of light through certain materials and the second is the reflection of light.
		\item Describe a specific use of a polarizing material.
	\end{enumerate}
	
	\item \textbf{Visible Spectra:} \\	
		Use the diffraction gratings to observe the spectra of the light sources in the large black box.
	\begin{enumerate}	
		\item Sketch the spectrum of each of the five light sources.
		\item What characteristic (if any) of each spectrum is "quantized?"  By quantized we mean that only certain values are observed.
		\item Why does each gas have a different spectrum?
		\item Briefly explain why the filament's spectrum is not quantized (continuous rather than discrete bands of color).
	\end{enumerate}
\end{enumerate}

\subsection {Photoelectric Effect}	
The photoelectric effect is the only observation in today's laboratory exercise that cannot be explained by the wave nature of electromagnetic radiation alone.  It can be explained by the existence of photons (your instructor should discuss the nature of photons or particles of light).
\begin{enumerate}
	\item Describe the operation of the photoelectric apparatus.
	\item Devise an experiment with red and blue filters that implies the photon energy of blue light is greater than the photon energy of red light.
	\item Pick some feature of the photoelectric effect that cannot be explained by waves alone and elaborate on how the wave picture is insufficient to describe this feature.
\end{enumerate}

	
\subsection {Ultraviolet Radiation (UV)}
 Ultraviolet radiation has higher frequencies and hence higher photon energies than visible light.  Many UV characteristics can be explained by the higher photon energy.  One example is the damage done to biological cells by ultraviolet radiation.
 \begin{enumerate}
	\item Observe fluorescence with the black light and the mineral samples provided.  Record the colors observed for each sample and use Figure 23.3 in your textbook to estimate the wavelength.
	\item Explain the process of fluorescence in terms of photon energies and atomic excitations and de-excitations.  Refer to Example 28.5 in your textbook.
	\item Using the estimated wavelengths, calculate an estimate for the energy of the photons emitted. 
	\item A weak source of UV is one with low intensity (a relatively small number of photons emitted per second).  Even weak sources of UV can be good sterilizers and pose a hazard to skin and other living tissue.  Explain why in terms of photon energy.
\end{enumerate}
	
\subsection {X-rays}
X-rays are produced when energetic electrons strike a material.  Inner shell electrons are ejected from some atoms when the incoming electrons strike them.  When another electron fills the hole left, the energy lost by the electron is emitted as an x-ray.  Cathode ray tubes (such as those in televisions and computer monitors) have energetic electrons that cause a screen to glow.  X-rays are also produced and must be shielded to protect the observer.  

Our cathode ray tube is not shielded so you can observe the x-rays it produces.  The intensity of the x-rays is too small to be harmful.
\begin{enumerate}
	\item Describe how you observe the x rays produced by the cathode ray tube.  Does their detection imply the existence of photons?  Explain.
	\item Where do the x-rays seem to originate?
	\item Demonstrate that the x-rays can be shielded against.
	\item How do the damaging properties of x-rays relate to their photon energy?
	\item What precautions are used to protect humans from x-rays produced by televisions and x-ray machines?  Note that all protection involves various combinations of shielding, distance, and time limitation of exposure.
\end{enumerate}

\section{Conclusions}
  For which types of electromagnetic radiation did you observe wave characteristics?  For which did you observe photon or particle characteristics?


  
%\endinput