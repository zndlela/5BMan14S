% !TEX root = 5b-main.tex
% 2013 Jan. Student labnotes

%--------------------------------------------------------------------------
\chapter{Laboratory Report Format} \label{a:rptformat}
\section{Laboratory and Results Overview}
Name (and Lab Partners):\hfill Date:\\
Section:. \hfill Day:\\

The report should also have a title which is usually the laboratory name.

\noindent \textbf{Lab Objective:} Write a single sentence or short paragraph, in your own words, stating the objective of the laboratory.\\
\textbf{Lab Overview:} Summarize in a single sentence or short paragraph the outcome of your lab experiment.
\begin{itemize}[itemsep=0pt]
	\item Often this will be a brief statement of how closely your observations or measurements agreed with or were consistent with predictions from a hypothesis, theory, or formula.
	\item Many labs involve observation of phenomena, so that your results might list or state your new understanding of the physical phenomena that you gained by doing the lab.
\end{itemize}

\section{Procedure}
\textbf{Equipment:} Provide a sketch of the setup of the major pieces of equipment and their arrangement.\\
\textbf{Experimental Procedure:} Summarize in a sentence or brief paragraph any exceptional procedures, techniques, or methods used during the lab.

\section{Data \& Analysis}
Report your data using tables, drawings, and graphs. Give an appropriate name or title to your tables, plots, and graphs. Example: Suppose you are measuring voltage across and current through a small bulb and record those values. A   poor title is, ``V vs. I'', or ``Voltage vs. Current''. A better title could be, ``Non-linear Resistance of a 12 V bulb''. Look in your text for examples of table and graph names. Be sure to include units for your table columns and include labels and units for the axes of all plots and graphs.

\section{Interpretation \& Conclusion}
Answer all questions required by the instructor and or posed in your lab manual. Where applicable or appropriate, comment on the accuracy and precision of your measurements, include your uncertainties, and identify possible sources of imprecision and uncertainty.

%--------------------------------------------------------------------------
\addtocounter{lab}{1}
\chapter{Physics 5: Lab Performance}
\section{Grading Rubric}
A few years ago, several instructors worked on revising aspects of the 5A course. One of the outcomes was a suggested rubric for grading some of the assignments in the lab and discussion sections. The rubric is reprinted below and may be of some use for labs and assignments where such grading is appropriate.
\begin{table*} \caption{Lab Performance Levels} \label{t:performance}
\centering
\begin{tabular}{p{2in}p{2in}p{2in}} \toprule
Score 9 - 10		&Score 8 - 9		&Score 7 - 8\\
\midrule
Student work in lab and in the report demonstrates an in depth understanding of scientific concepts and their applications.
	&Student work in lab and in the report demonstrates an in depth understanding of scientific concepts and their applications.
	&Student work in lab and in the report demonstrates an in depth understanding of scientific concepts and their applications.\\
	\midrule
All work is clear and complete and exceeds expectations. It demonstrates skill in formulating strategies for measurements and observations, and the ability to make logical and appropriate arguments.
	&The work is generally clear and complete and meets expectations. It shows evidence of skill in formulating strategies for measurements and observations, and the ability to make logical and appropriate arguments
	&Most work is clear and complete but does not meet all the expectations. It shows some skill in formulating strategies for measurements and observations, and the ability to make logical and appropriate arguments\\
	\midrule
Discussions make strong connections to events outside the laboratory environment including personal experiences and or topics in other scientific disciplines.
	&Discussions establish adequate connections to events outside the laboratory environment including personal experiences and or topics in other scientific disciplines
	&Discussions establish some connections to events outside the laboratory environment including personal experiences and or topics in other scientific disciplines\\

\bottomrule
	
Score 6 - 7
&Score 5 - 6\\ \midrule

Student work in lab and in the report demonstrates limited understanding and application of scientific concepts.
	&Student work in lab and in the report demonstrates little or no understanding and application of scientific concepts.\\
	\midrule
Work is incomplete and not presented clearly. It shows limited skill in formulating strategies for measurements and observations, and the ability to make logical and appropriate arguments.
	&Work shows almost no progress toward completion and presents vague and irrelevant information. It shows little or no skill in formulating strategies for measurements and observations, and the ability to make logical and appropriate arguments.\\
	\midrule
Discussions make minimal or no connections to events outside the laboratory environment including personal experiences and or topics in other scientific disciplines
	&Discussions do not establish connections to events outside the laboratory environment including personal experiences and or topics in other scientific disciplines.\\
	\bottomrule
\end{tabular}
\end{table*}


%\end{input}
