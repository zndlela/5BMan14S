%!TEX root = 5b-main.tex

% tufte-book loads hyperref with its own options

\IfFileExists{bergamo.sty}{\usepackage[osf]{bergamo}}{}% Bembo
\IfFileExists{chantill.sty}{\usepackage{chantill}}{}% Gill Sans

%%
% For drawing nice pictures
\usepackage{tikz}
\usetikzlibrary{calc}

%%
% For circuit diagrams (some)
\usepackage[siunitx]{circuitikz}

%%
% For Mathematic work
\usepackage{amsmath}

%%
% For human readable calculations
\usepackage{units}

%%
% For nicely typeset tabular material
\usepackage{booktabs}

%%
% For resuming interrupted lists
\usepackage{enumitem}

%%
% Use the textmode font for mathematics
\usepackage[italic]{mathastext}

%%
% Rotate some of our floats
\usepackage{rotating}

%%
% For graphics / images
\usepackage{graphicx}
	\setkeys{Gin}{width=\linewidth,totalheight=\textheight,keepaspectratio}
	\graphicspath{{graphics/}{images/}}

% The fancyvrb package lets us customize the formatting of verbatim
% environments.  We use a slightly smaller font.
\usepackage{fancyvrb}
	\fvset{fontsize=\normalsize}

% todo notes and short comments
\usepackage[textsize=footnotesize, % place todonote items in text
%	background color=green,
%	disable	% prevent display of all notes
]{todonotes}
% syntax: 
%  \todo{inserts a plain todo note}
%  \todo[color=blue!40]{inserts note colored 40 % blue.}
%  \todo[nolist]{show note in margin, but not in list of todos.}
%  \todo[size=\small, color=green!40]{A note with a small fontsize, colored 40% green.}
%  \todo[inline]{place note on its own line}
%  \todo[noline]{note in margin, does not point to text.}

% Prints the month name (e.g., January) and the year (e.g., 2008)
\newcommand{\monthyear}{%
  \ifcase\month\or January\or February\or March\or April\or May\or June\or
  July\or August\or September\or October\or November\or
  December\fi\space\number\year
}

% Prints an epigraph and speaker in sans serif, all-caps type.
\newcommand{\openepigraph}[2]{%
  %\sffamily\fontsize{14}{16}\selectfont
  \begin{fullwidth}
  \sffamily\large
  \begin{doublespace}
  \noindent\allcaps{#1}\\% epigraph
  \noindent\allcaps{#2}% author
  \end{doublespace}
  \end{fullwidth}
}

% Inserts a blank page
\newcommand{\blankpage}{\newpage\hbox{}\thispagestyle{empty}\newpage}

\newcommand{\degree}{\ensuremath{^{\circ}}}
\newcommand{\slashfrac}[2]{\ensuremath{\left.#1\middle/#2\right.}}
\newcommand{\rref}[1]{Eq. \ref{#1}}
\newcommand{\fref}[1]{Fig. \ref{#1}}
\newcommand{\tref}[1]{Tab. \ref{#1}}


